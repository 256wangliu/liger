\documentclass[]{article}
\usepackage{lmodern}
\usepackage{amssymb,amsmath}
\usepackage{ifxetex,ifluatex}
\usepackage{fixltx2e} % provides \textsubscript
\ifnum 0\ifxetex 1\fi\ifluatex 1\fi=0 % if pdftex
  \usepackage[T1]{fontenc}
  \usepackage[utf8]{inputenc}
\else % if luatex or xelatex
  \ifxetex
    \usepackage{mathspec}
  \else
    \usepackage{fontspec}
  \fi
  \defaultfontfeatures{Ligatures=TeX,Scale=MatchLowercase}
\fi
% use upquote if available, for straight quotes in verbatim environments
\IfFileExists{upquote.sty}{\usepackage{upquote}}{}
% use microtype if available
\IfFileExists{microtype.sty}{%
\usepackage{microtype}
\UseMicrotypeSet[protrusion]{basicmath} % disable protrusion for tt fonts
}{}
\usepackage[margin=1in]{geometry}
\usepackage{hyperref}
\hypersetup{unicode=true,
            pdftitle={Comparing and contrasting heterogeneous single cell profiles using Analogizer},
            pdfauthor={Joshua D. Welch},
            pdfborder={0 0 0},
            breaklinks=true}
\urlstyle{same}  % don't use monospace font for urls
\usepackage{color}
\usepackage{fancyvrb}
\newcommand{\VerbBar}{|}
\newcommand{\VERB}{\Verb[commandchars=\\\{\}]}
\DefineVerbatimEnvironment{Highlighting}{Verbatim}{commandchars=\\\{\}}
% Add ',fontsize=\small' for more characters per line
\usepackage{framed}
\definecolor{shadecolor}{RGB}{248,248,248}
\newenvironment{Shaded}{\begin{snugshade}}{\end{snugshade}}
\newcommand{\KeywordTok}[1]{\textcolor[rgb]{0.13,0.29,0.53}{\textbf{#1}}}
\newcommand{\DataTypeTok}[1]{\textcolor[rgb]{0.13,0.29,0.53}{#1}}
\newcommand{\DecValTok}[1]{\textcolor[rgb]{0.00,0.00,0.81}{#1}}
\newcommand{\BaseNTok}[1]{\textcolor[rgb]{0.00,0.00,0.81}{#1}}
\newcommand{\FloatTok}[1]{\textcolor[rgb]{0.00,0.00,0.81}{#1}}
\newcommand{\ConstantTok}[1]{\textcolor[rgb]{0.00,0.00,0.00}{#1}}
\newcommand{\CharTok}[1]{\textcolor[rgb]{0.31,0.60,0.02}{#1}}
\newcommand{\SpecialCharTok}[1]{\textcolor[rgb]{0.00,0.00,0.00}{#1}}
\newcommand{\StringTok}[1]{\textcolor[rgb]{0.31,0.60,0.02}{#1}}
\newcommand{\VerbatimStringTok}[1]{\textcolor[rgb]{0.31,0.60,0.02}{#1}}
\newcommand{\SpecialStringTok}[1]{\textcolor[rgb]{0.31,0.60,0.02}{#1}}
\newcommand{\ImportTok}[1]{#1}
\newcommand{\CommentTok}[1]{\textcolor[rgb]{0.56,0.35,0.01}{\textit{#1}}}
\newcommand{\DocumentationTok}[1]{\textcolor[rgb]{0.56,0.35,0.01}{\textbf{\textit{#1}}}}
\newcommand{\AnnotationTok}[1]{\textcolor[rgb]{0.56,0.35,0.01}{\textbf{\textit{#1}}}}
\newcommand{\CommentVarTok}[1]{\textcolor[rgb]{0.56,0.35,0.01}{\textbf{\textit{#1}}}}
\newcommand{\OtherTok}[1]{\textcolor[rgb]{0.56,0.35,0.01}{#1}}
\newcommand{\FunctionTok}[1]{\textcolor[rgb]{0.00,0.00,0.00}{#1}}
\newcommand{\VariableTok}[1]{\textcolor[rgb]{0.00,0.00,0.00}{#1}}
\newcommand{\ControlFlowTok}[1]{\textcolor[rgb]{0.13,0.29,0.53}{\textbf{#1}}}
\newcommand{\OperatorTok}[1]{\textcolor[rgb]{0.81,0.36,0.00}{\textbf{#1}}}
\newcommand{\BuiltInTok}[1]{#1}
\newcommand{\ExtensionTok}[1]{#1}
\newcommand{\PreprocessorTok}[1]{\textcolor[rgb]{0.56,0.35,0.01}{\textit{#1}}}
\newcommand{\AttributeTok}[1]{\textcolor[rgb]{0.77,0.63,0.00}{#1}}
\newcommand{\RegionMarkerTok}[1]{#1}
\newcommand{\InformationTok}[1]{\textcolor[rgb]{0.56,0.35,0.01}{\textbf{\textit{#1}}}}
\newcommand{\WarningTok}[1]{\textcolor[rgb]{0.56,0.35,0.01}{\textbf{\textit{#1}}}}
\newcommand{\AlertTok}[1]{\textcolor[rgb]{0.94,0.16,0.16}{#1}}
\newcommand{\ErrorTok}[1]{\textcolor[rgb]{0.64,0.00,0.00}{\textbf{#1}}}
\newcommand{\NormalTok}[1]{#1}
\usepackage{graphicx,grffile}
\makeatletter
\def\maxwidth{\ifdim\Gin@nat@width>\linewidth\linewidth\else\Gin@nat@width\fi}
\def\maxheight{\ifdim\Gin@nat@height>\textheight\textheight\else\Gin@nat@height\fi}
\makeatother
% Scale images if necessary, so that they will not overflow the page
% margins by default, and it is still possible to overwrite the defaults
% using explicit options in \includegraphics[width, height, ...]{}
\setkeys{Gin}{width=\maxwidth,height=\maxheight,keepaspectratio}
\IfFileExists{parskip.sty}{%
\usepackage{parskip}
}{% else
\setlength{\parindent}{0pt}
\setlength{\parskip}{6pt plus 2pt minus 1pt}
}
\setlength{\emergencystretch}{3em}  % prevent overfull lines
\providecommand{\tightlist}{%
  \setlength{\itemsep}{0pt}\setlength{\parskip}{0pt}}
\setcounter{secnumdepth}{0}
% Redefines (sub)paragraphs to behave more like sections
\ifx\paragraph\undefined\else
\let\oldparagraph\paragraph
\renewcommand{\paragraph}[1]{\oldparagraph{#1}\mbox{}}
\fi
\ifx\subparagraph\undefined\else
\let\oldsubparagraph\subparagraph
\renewcommand{\subparagraph}[1]{\oldsubparagraph{#1}\mbox{}}
\fi

%%% Use protect on footnotes to avoid problems with footnotes in titles
\let\rmarkdownfootnote\footnote%
\def\footnote{\protect\rmarkdownfootnote}

%%% Change title format to be more compact
\usepackage{titling}

% Create subtitle command for use in maketitle
\newcommand{\subtitle}[1]{
  \posttitle{
    \begin{center}\large#1\end{center}
    }
}

\setlength{\droptitle}{-2em}
  \title{Comparing and contrasting heterogeneous single cell profiles using
Analogizer}
  \pretitle{\vspace{\droptitle}\centering\huge}
  \posttitle{\par}
  \author{Joshua D. Welch}
  \preauthor{\centering\large\emph}
  \postauthor{\par}
  \predate{\centering\large\emph}
  \postdate{\par}
  \date{2018-04-20}


\begin{document}
\maketitle

\subsection{Data Preprocessing}\label{data-preprocessing}

\begin{Shaded}
\begin{Highlighting}[]
\NormalTok{dge1 =}\StringTok{ }\KeywordTok{readRDS}\NormalTok{(}\StringTok{"dge1.RDS"}\NormalTok{) }\CommentTok{#genes in rows, cells in columns, rownames and colnames included. Sparse matrix format not accepted.}
\NormalTok{dge2 =}\StringTok{ }\KeywordTok{readRDS}\NormalTok{(}\StringTok{"dge2.RDS"}\NormalTok{)}
\NormalTok{analogy =}\StringTok{ }\KeywordTok{Analogizer}\NormalTok{(}\KeywordTok{list}\NormalTok{(}\DataTypeTok{name1=}\NormalTok{dge1,}\DataTypeTok{name2=}\NormalTok{dge2)) }\CommentTok{#Can also pass in more than 2 datasets}
\NormalTok{analogy =}\StringTok{ }\KeywordTok{normalize}\NormalTok{(analogy)}
\NormalTok{analogy =}\StringTok{ }\KeywordTok{selectGenes}\NormalTok{(analogy,}\DataTypeTok{varthresh =} \FloatTok{0.2}\NormalTok{)}
\NormalTok{analogy =}\StringTok{ }\KeywordTok{scaleNotCenter}\NormalTok{(analogy)}
\end{Highlighting}
\end{Shaded}

\subsection{Performing the
Factorization}\label{performing-the-factorization}

\begin{Shaded}
\begin{Highlighting}[]
\NormalTok{analogy =}\StringTok{ }\KeywordTok{optimizeALS}\NormalTok{(analogy,}\DataTypeTok{k=}\DecValTok{10}\NormalTok{) }\CommentTok{#nrep=1 is the default and gives a good idea of the results. Recommend more than one initialization (nrep=10) for final analyses}
\NormalTok{analogy =}\StringTok{ }\KeywordTok{quantile_norm}\NormalTok{(analogy)}
\NormalTok{analogy =}\StringTok{ }\KeywordTok{clusterLouvainJaccard}\NormalTok{(analogy) }\CommentTok{#Optional clustering step to refine the clusters from the factorization}
\end{Highlighting}
\end{Shaded}

\subsection{Visualizing the results}\label{visualizing-the-results}

\begin{Shaded}
\begin{Highlighting}[]
\NormalTok{analogy =}\StringTok{ }\KeywordTok{run_tSNE}\NormalTok{(analogy)}
\KeywordTok{plotByDatasetAndCluster}\NormalTok{(analogy) }\CommentTok{#Can also pass in different set of cluster labels to plot}
\KeywordTok{pdf}\NormalTok{(}\StringTok{"word_clouds.pdf"}\NormalTok{)}
\KeywordTok{plot_word_clouds}\NormalTok{(analogy)}
\KeywordTok{dev.off}\NormalTok{()}
\end{Highlighting}
\end{Shaded}

\subsection{Updating the
Factorization}\label{updating-the-factorization}

\begin{Shaded}
\begin{Highlighting}[]
\NormalTok{analogy =}\StringTok{ }\KeywordTok{optimizeNewK}\NormalTok{(analogy,}\DataTypeTok{k=}\DecValTok{15}\NormalTok{) }\CommentTok{#Can also decrease K}
\NormalTok{analogy =}\StringTok{ }\KeywordTok{optimizeNewData}\NormalTok{(analogy,}\DataTypeTok{newdata=}\KeywordTok{list}\NormalTok{(}\DataTypeTok{name1=}\NormalTok{dge1.new,}\DataTypeTok{name2=}\NormalTok{dge2.new),}\DataTypeTok{which.datasets=}\KeywordTok{list}\NormalTok{(name1,name2),}\DataTypeTok{add.to.existing=}\NormalTok{T) }\CommentTok{#Add new batches from the same condition/technology/species/protocol}
\NormalTok{analogy =}\StringTok{ }\KeywordTok{optimizeNewData}\NormalTok{(analogy,}\DataTypeTok{newdata=}\KeywordTok{list}\NormalTok{(}\DataTypeTok{name3=}\NormalTok{dge3,}\DataTypeTok{name4=}\NormalTok{dge4),}\DataTypeTok{which.datasets=}\KeywordTok{list}\NormalTok{(name1,name2),}\DataTypeTok{add.to.existing=}\NormalTok{F) }\CommentTok{#Add completely new datasets. Specify which existing datasets are most similar.}
\NormalTok{analogy =}\StringTok{ }\KeywordTok{optimizeSubset}\NormalTok{(analogy,cell.subset) }\CommentTok{#cell.subset is a list of cells to retain from each dataset}
\end{Highlighting}
\end{Shaded}


\end{document}
